\section{Introduction} % what is the problem. what is our solution.

Tuning applications could be considered an art form. It requires expertise in identifying important underlying factors and understanding their correlation with the performance. Auto-tuning is an attempt to find such well-performing configurations automatically. Since the search space is often too large to test exhaustively, machine learning is utilized to reduce time to solution, mostly by predicting or modeling program performance for each configuration instead of actually running it.

As of current, althought there are many researches on using machine learning in auto-tuning, not many of them cover architectures other than CPUs (Central Processing Units). In this project, I compare how well selected machine learning algorithms assist tuning a stencil computation application on GPU. Selected algorithms are Gradient Boosted Regression Tree, Kernel Ridge Regression (KRR), Support Vector Machine (SVM), and Kernel Canonical Correlation Analysis (KCCA).

% How well they perform against each other
