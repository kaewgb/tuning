\section{Introduction} % what is the problem. what is our solution.

Tuning applications could be considered an art form. It requires expertise in identifying important underlying factors and understanding their correlation with the performance. Auto-tuning is an attempt to find such well-performing configurations automatically. Since the search space is often too large to test exhausibly, machine learning is well-adopted in this field to reduce time to solution, mostly by predicting or modeling program performance for each configuration instead of actually running it.

As of current, althought there are many researches on using machine learning in auto-tuning, not many of them cover architectures other than CPUs (Central Processing Units). In addition, even with the existing frameworks for CPU ones are considerably diverge and are more of studying a certain algorithm on a certain application. This leaves an open question of what algorithm to chose for each application. In this project, I compare how well selected machine learning algorithms assist tuning a stencil computation application on GPU. Selected algorithms are Gradient Boosted Regression Tree, Kernel Ridge Regression (KRR), Support Vector Machine (SVM), and Kernel Canonical Correlation Analysis (KCCA).

% How well they perform against each other
